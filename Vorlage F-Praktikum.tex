% Vorlage zur Erstellung eines Praktikumsprotokolls
% Version: 2017-10-05
%
% Diese Vorlage soll als Grundlage für die Erstellung eines Protokolls
% im Rahmen des Praktikums (F-Praktikum Physik, TU Dresden) dienen.
% Viele wichtige Pakete sind schon mit eingebunden, aber es können
% auch noch weiter Pakete notwendig werden. Auf jedene Fall ist zu
% beachten, dass dies keine offizielle Vorlage ist und somit auch
% eine andere Form gewählt werden kann und evtl. auch von den
% Betreuern erwartet wird.
%
% Erstellt wird das Dokument mit dem Befehl "pdflatex Protokoll", wenn
% ihr das Dokument per Kommandozeile erstellen wollt. Ihr müsst euch dafür
% im Verzeichnis des Protokolls befinden. Diejenigen, die mit einem
% LaTeX-Editor schreiben, stellen als Compiler entsprechend PDFLaTeX ein.
%
% Bitte schaut auch mindest alle Stellen an, die mit TODO gekennzeichnet sind.
%
% Zusätzlich zur Präambel ist auch noch etwas Beispieltext mit angegeben. Der
% die Verwendung einiger der etlichen Befehle zeigt.
% Das Bibliographieverzeichnis ist manuell erstellt worden, was für ein
% Praktikum meistens der weniger aufwändige Weg ist.
% Für umfangreichere Verzeichnisse hilft es sich mit „biblatex“ vertraut zu
% machen
%
% Die Vorlage kann *beliebig* weiterverwendet werden.
% CC0 Henning Iseke <h_i_@online.de>
%---------------------------------------------------------------------

% -- Anfang Präambel
\documentclass[german,  % Standardmäßig deutsche Eigenarten, englisch -> english
parskip=full,  % Absätze durch Leerzeile trennen
%bibliography=totoc,  % Literatur im Inhaltsverzeichnis (ist unüblich)
%draft,  % TODO: Entwurfsmodus -> entfernen für endgültige Version
]{scrartcl}
\usepackage[utf8]{inputenc}  % Kodierung der Datei
\usepackage[T1]{fontenc}  % Vollen Umfang der Schriftzeichen
\usepackage[ngerman]{babel}  % Sprache auf Deutsch (neue Rechtschreibung)

% Mathematik und Größen
\usepackage{amsmath}
\usepackage[locale=DE,  % deutsche Eigenarten, englisch -> US
separate-uncertainty,  % Unsicherheiten seperat angeben (mit ±)
]{siunitx}
\usepackage{physics}  % Erstellung von Gleichungen vereinfachen

\usepackage{graphicx}  % Bilder einbinden \includegraphics{Pfad/zur/Datei(ohne Dateiendung)}

% Gestaltung
%\usepackage{microtype}  % Mikrotypographie (kann man am Ende verwenden)
\usepackage{booktabs}  % schönere Tabellen
\usepackage[toc]{multitoc}  % mehrspaltiges Inhaltsverzeichnis
\usepackage{csquotes}  % Anführungszeichen mit \enquote
\usepackage{caption}  % Anpassung der Bildunterschriften, Tabellenüberschriften
\usepackage{subcaption}  % Unterabbildungen, Untertabellen, …
\usepackage{enumitem}  % Listen anpassen
\setlist{itemsep=-10pt}  % Abstände zwischen Listenpunkten verringern

% Manipulation des Seitenstils
\usepackage{scrpage2}
% Kopf-/Fußzeilen setzen
\pagestyle{scrheadings}  % Stil für die Seite setzen
\clearscrheadings  % Stil zurücksetzen, um ihn neu zu definieren
\automark{section}  % Abschnittsnamen als Seitenbeschriftung verwenden
\ofoot{\pagemark}  % Seitenzahl außen in Fußzeile
\ihead{\headmark}  % Seitenbeschriftung mittig in Kopfzeile
\setheadsepline{.5pt}  % Kopzeile durch Linie abtrennen

\usepackage[hidelinks]{hyperref}  % Links und weitere PDF-Features

% TODO: Titel und Autor, … festlegen
\newcommand*{\titel}{<Titel des Versuchs>}
\newcommand*{\autor}{<Autoren>}
\newcommand*{\abk}{<Abkürzung des Versuchs>}
\newcommand*{\betreuer}{<Betreuer>}
\newcommand*{\messung}{<Datum>}
\newcommand*{\ort}{<Ort>}

\hypersetup{pdfauthor={\autor}, pdftitle={\titel}}  % PDF-Metadaten setzen

% automatischen Titel konfigurieren
\titlehead{F-Praktikum \abk \hfill TU Dresden}
\subject{Versuchsprotokoll}
\title{\titel}
\author{\autor}
\date{\begin{tabular}{ll}
Protokoll: & \today\\
Messung: & \messung\\
Ort: & \ort\\
Betreuer: & \betreuer\end{tabular}}

% -- Ende Präambel

\begin{document}
\begin{titlepage}
\maketitle  % Titel setzen
\tableofcontents  % Inhaltsverzeichnis setzen
\end{titlepage}

% ----- BEISPIELTEXT ANFANG -----

    \section{Einleitung}
    Der Versuch hat das Ziel, sich mit der Dosimetrie vertraut zu machen.
    Sie ist notwendig, da sich die Energieübertragung bei komplizierten
    Versuchsaufbauten nur sehr aufwendig berechnen/simulieren lässt.
    Insbesondere liefert sie eine wichtige Größe, um etwas über die durch
    Wechselwirkung von Strahlung auf Materie übertragene Energie auszusagen.

    Die Dosimetrie ist damit ein sehr wichtiger Teil vieler Versuche, die sich
    mit Strahlung beschäftigen, auch um Gefährdungen zu minimieren bzw.
    bestenfalls zu eliminieren.

    \section{Theorie}
    \subsection{A-Regeln}
    Grundsätzlich gilt es natürlich die \enquote{A-Regeln}
    \cite{wiki:Praktischer_Strahlenschutz} einzuhalten d.\,h.:
    \begin{description}
    \item[Aktivität] begrenzen
    \item[Aufenthaltsdauer] minimieren
    \item[Abstand] halten
    \item[Abschirmung] verwenden
    \item[Aufnahme] vermeiden
    \item[Ausbildung] im Umgang mit Strahlung
    \end{description}
    denn dadurch lässt sich auf einfache Weise das Risiko beim Umgang mit
    Strahlenquellen minimieren.

    Um die Energieübertragung bzw. Belastungen für den menschlichen
    Körper beim Umgang mit Strahlenquellen in Folge von Wechselwirkungen zu
    quantifizieren gibt es drei wichtige Größen.

    \subsection{Größen}
    Die gemessene Größe ist die \textbf{Energiedosis}
    %
    \begin{equation}
    D = \dv{E}{m},
    \end{equation}
    %
    die in der
    Einheit \textbf{Gray} ($\si{\gray} = \si{\joule\per\kg}$) angegeben wird.
    Sie dient zur Angabe der Energie ($\mathrm{d}E$) die pro
    Massenelement ($\mathrm{d}m$) an einem bestimmten Ort deponiert wird.
    Über Wichtungsfaktoren für die Strahlungsart ($w_\mathrm{R}$) erhält man
    die \textbf{Äquivalentdosis}
    %
    \begin{equation}
    H_\mathrm{T} = w_\mathrm{R}\cdot D_\mathrm{T}
    \end{equation}
    %
    für ein Organ $\mathrm{T}$, hierbei wird die unterschiedliche Wirkung der
    verschiedenen Strahlungsarten berücksichtigt.

    Die \textbf{effektive Dosis} ($H_\text{E}$),
    die in der
    Einheit \textbf{Sievert} ($\si{\sievert}=\si{\joule\per\kilogram}$)
    angegeben wird, berechnet sich aus der Äquivalentdosis. Bei der Berechnung
    wird neben der Art der Strahlung auch die Empfindlichkeit des bestrahlten
    Organs mit einem Korrekturfaktor $w_\mathrm{T}$ mit berücksichtigt.
    Schlussendlich lässt sich die effektive Dosis über
    %
    \begin{equation}
    H_\mathrm{E}
    = \sum\limits_\mathrm{T} w_\mathrm{T}\cdot w_\mathrm{R}\cdot D_\mathrm{T}
    \end{equation}
    %
    berechnen, die Wichtungsfaktoren findet man u.\,a. in
    \cite[Anlage VI]{strlschv}.
    
    \begin{figure}[htbp]
    \centering
    \fbox{
    \begin{subfigure}{.45\textwidth}
    \fbox{TEST1}
    \caption{Test 1.}\label{fig:test1}
    \end{subfigure}
    }
    \begin{subfigure}{.45\textwidth}
    \fbox{TEST2}
    \caption{Test 2.}\label{fig:test2}
    \end{subfigure}
    \caption{Test 1+2: \subref{fig:test1} und \subref{fig:test2}.}
    \label{fig:test}
    \end{figure}
    
    Es ist nicht wirklich interessant sich Abbildung~\ref{fig:test} anzuschauen,
    da sich Unterabbildung~\ref{fig:test1} und \ref{fig:test2} nur marginal
    unterscheiden.

    % Bibliographie/Literaturverzeichnis
    \begin{thebibliography}{9}
    \bibitem{strlschv}
    StrlSchV (2001),
    \emph{Verordnung über den Schutz vor Schäden durch ionisierende Strahlen},
    \url{http://www.gesetze-im-internet.de/strlschv_2001/index.html},
    5.\,Nov.~2015.
    \bibitem{wiki:Praktischer_Strahlenschutz}
    Wikipedia,
    \emph{Praktischer Strahlenschutz},
    \url{https://de.wikipedia.org/wiki/Strahlenschutz#Praktischer_Strahlenschutz},
    5.\,Nov.~2015.
    \end{thebibliography}

% ----- BEISPIELTEXT ENDE -----

\end{document}